\documentclass[12pt,a4paper]{article}

% Pakiety
\usepackage[utf8]{inputenc}
\usepackage[T1]{fontenc}
\usepackage[polish]{babel}
\usepackage{geometry}
\geometry{margin=2.5cm}
\usepackage{graphicx}
\usepackage{booktabs}
\usepackage{amsmath}
\usepackage{hyperref}
\usepackage{caption}
\usepackage{subcaption}
\usepackage{float}
\usepackage{enumitem}
\usepackage{csquotes}
\usepackage[style=apa,backend=biber]{biblatex}
\addbibresource{bibliografia.bib}

\usepackage{setspace}
\onehalfspacing

% -------------------------------------------------------------------
\title{Wpływ kompresji obrazu na skuteczność modeli głębokiego uczenia w diagnostyce kardiologicznej}
\author{}
\date{}

\begin{document}
\maketitle

% ===================================================================
% 1. WSTĘP (Abstract / Wprowadzenie)
% ===================================================================
\section{Wstęp}

Obrazowanie medyczne stanowi fundament współczesnej diagnostyki kardiologicznej. Wraz ze wzrostem liczby wykonywanych badań oraz rozdzielczości obrazów, rośnie zapotrzebowanie na efektywne metody przechowywania i przesyłania danych. Kompresja obrazu jest powszechnie stosowanym rozwiązaniem tego problemu, jednak jej wpływ na skuteczność automatycznych systemów diagnostycznych opartych na głębokim uczeniu pozostaje niedostatecznie zbadany.

W~niniejszej pracy zbadano wpływ trzech formatów kompresji -- JPEG, JPEG2000 oraz AVIF -- na dokładność modeli klasyfikacji segmentów tętnic wieńcowych w~obrazach angiografii rentgenowskiej. Przeprowadzono dwa komplementarne eksperymenty: trening na danych skompresowanych z~ewaluacją na oryginałach (Eksperyment~A) oraz trening na oryginałach z~ewaluacją na danych skompresowanych (Eksperyment~B). Badania wykonano na zbiorze danych ARCADE zawierającym 3000 ekspertowo zanotowanych obrazów.

% ===================================================================
% 2. MOTYWACJA I KONTEKST PRACY
% ===================================================================
\section{Motywacja i kontekst pracy}

Systemy archiwizacji obrazów medycznych (PACS -- \textit{Picture Archiving and Communication System}) oraz platformy telemedyczne wymagają kompresji danych w celu redukcji kosztów przechowywania i~przyspieszenia transmisji. Standard DICOM (\textit{Digital Imaging and Communications in Medicine}) dopuszcza stosowanie kompresji stratnej, w~tym formatu JPEG2000, jednakże brak jest jednoznacznych wytycznych dotyczących dopuszczalnego poziomu kompresji w~kontekście automatycznej analizy AI.

Jednocześnie modele głębokiego uczenia coraz częściej wspomagają diagnostykę chorób tętnic wieńcowych (CAD -- \textit{Coronary Artery Disease}), które pozostają główną przyczyną zgonów na świecie. Angiografia rentgenowska jest złotym standardem w~diagnostyce CAD, a~automatyczna klasyfikacja segmentów naczyniowych według schematu SYNTAX Score umożliwia obiektywną ocenę stopnia zaawansowania choroby.

Pojawienie się nowoczesnego formatu AVIF (opartego na kodeku AV1, 2019) otwiera nowe możliwości kompresji, lecz jego przydatność w~obrazowaniu medycznym nie została dotychczas zbadana. Niniejsza praca wypełnia tę lukę, oferując pierwsze kompleksowe porównanie trzech formatów kompresji w~kontekście ich wpływu na modele AI stosowane w~kardiologii.

% ===================================================================
% 3. CELE I ZAŁOŻENIA (HIPOTEZA)
% ===================================================================
\section{Cele i założenia}

\subsection{Cel główny}
Określenie wpływu kompresji stratnej obrazów angiografii rentgenowskiej na skuteczność modeli głębokiego uczenia w~zadaniu klasyfikacji segmentów tętnic wieńcowych.

\subsection{Cele szczegółowe}
\begin{enumerate}
    \item Porównanie trzech formatów kompresji (JPEG, JPEG2000, AVIF) pod kątem jakości obrazu mierzonej wskaźnikami PSNR i~SSIM.
    \item Zbadanie wpływu kompresji danych treningowych na dokładność modelu (Eksperyment~A).
    \item Zbadanie odporności modelu wytrenowanego na danych oryginalnych na kompresję danych wejściowych (Eksperyment~B).
    \item Wyznaczenie optymalnego poziomu kompresji zapewniającego maksymalną redukcję rozmiaru przy minimalnej utracie dokładności klasyfikacji.
    \item Sformułowanie rekomendacji dla systemów PACS i~platform telemedycznych.
\end{enumerate}

\subsection{Hipoteza}
Postawiono następujące hipotezy badawcze:
\begin{enumerate}
    \item[H1:] Format JPEG2000, będący standardem DICOM, zapewnia lepszą zachowanie skuteczności modeli AI niż klasyczny JPEG przy porównywalnym stopniu kompresji.
    \item[H2:] Format AVIF umożliwia osiągnięcie wyższego stopnia kompresji niż JPEG i~JPEG2000 przy zachowaniu porównywalnej skuteczności modeli.
    \item[H3:] Istnieje progowy poziom kompresji, poniżej którego następuje gwałtowny spadek dokładności klasyfikacji, niezależnie od zastosowanego formatu.
\end{enumerate}

% ===================================================================
% 4. ZAWARTOŚĆ PRACY
% ===================================================================
\section{Zawartość pracy}

Praca składa się z~następujących rozdziałów. W~rozdziale~\ref{sec:literatura} przedstawiono przegląd literatury dotyczącej kompresji obrazów medycznych oraz zastosowań głębokiego uczenia w~angiografii wieńcowej. Rozdział~\ref{sec:materialy} zawiera opis zastosowanych metod, modeli i~mierników skuteczności. W~rozdziale~\ref{sec:eksperyment} opisano przebieg eksperymentów, strukturę danych oraz kolejne etapy badania. Rozdział~\ref{sec:wyniki} prezentuje uzyskane wyniki w~formie tabel i~wykresów wraz z~ich interpretacją. W~rozdziale~\ref{sec:wnioski} sformułowano wnioski, a~w~rozdziale~\ref{sec:podsumowanie} dokonano podsumowania pracy.

% ===================================================================
% 5. PRZEGLĄD LITERATURY
% ===================================================================
\section{Przegląd literatury}
\label{sec:literatura}

\subsection{Kompresja obrazów medycznych}

Kompresja obrazów medycznych jest przedmiotem badań od lat 90. XX~wieku. Wyróżnia się kompresję bezstratną (zachowującą pełną informację) oraz stratną (wprowadzającą nieodwracalne zmiany). W~systemach PACS powszechnie stosowany jest format JPEG2000, który w~roku 2003 został włączony do standardu DICOM jako preferowany format kompresji stratnej \parencite{dicom2024}.

Format JPEG (Joint Photographic Experts Group) wykorzystuje dyskretną transformatę kosinusową (DCT) operującą na blokach 8$\times$8 pikseli, co przy niskich poziomach jakości prowadzi do charakterystycznych artefaktów blokowych. Format JPEG2000 stosuje transformatę falkową (DWT -- \textit{Discrete Wavelet Transform}), eliminując artefakty blokowe i~zapewniając lepszą jakość przy niższych przepływnościach \parencite{taubman2002jpeg2000}.

AVIF (\textit{AV1 Image File Format}) jest nowoczesnym formatem kompresji obrazu, opartym na kodeku wideo AV1, opublikowanym w~2019 roku. Wykorzystuje zaawansowane techniki predykcji wewnątrzramkowej i~transformaty, oferując znacząco lepszą efektywność kompresji w~porównaniu z~formatami poprzedniej generacji \parencite{avif2019}.

\subsection{Zbiór danych ARCADE}

ARCADE (\textit{Automatic Region-based Coronary Artery Disease diagnostics using x-ray angiography imagEs}) jest publicznie dostępnym zbiorem danych opublikowanym w~ramach wyzwania MICCAI 2023. Zbiór zawiera 3000 ekspertowo zanotowanych obrazów angiografii rentgenowskiej podzielonych na dwa podzbiory: 1500 obrazów do klasyfikacji segmentów naczyniowych według schematu SYNTAX Score (26 klas) oraz 1500 obrazów do detekcji zwężeń \parencite{popov2024arcade}.

\subsection{Zastosowania głębokiego uczenia w angiografii wieńcowej}

W~ostatnich latach opublikowano szereg prac wykorzystujących zbiór ARCADE do różnych zadań diagnostycznych. Framework LASF oparty na YOLOv8 osiągnął wysoką skuteczność w~segmentacji naczyń wieńcowych, przewyższając architektury U-Net i~DeepLabV3Plus \parencite{lasf2025}. Badanie porównawcze architektur detekcji obiektów (Grounding DINO, YOLO, DINO-DETR) wykazało zróżnicowaną skuteczność w~detekcji zwężeń \parencite{stenosis2025}. Model UCNet oparty na warunkowej sieci generatywnej (cGAN) osiągnął średni F1 score 84,43\% w~klasyfikacji 20 segmentów tętnic wieńcowych \parencite{ucnet2025}. Zastosowanie grafowych sieci konwolucyjnych do reprezentacji struktury drzewa wieńcowego pozwoliło na osiągnięcie F1 score 53,68 \parencite{graphcnn2025}.

Żadna z~dotychczasowych prac nie badała jednak wpływu kompresji obrazu na skuteczność modeli klasyfikacji w~angiografii wieńcowej, co stanowi istotną lukę badawczą.

% ===================================================================
% 6. MATERIAŁY I METODY
% ===================================================================
\section{Materiały i metody}
\label{sec:materialy}

\subsection{Formaty kompresji}

\subsubsection{JPEG}
Format JPEG wykorzystuje dyskretną transformatę kosinusową (DCT) stosowaną na blokach 8$\times$8 pikseli. Współczynniki DCT podlegają kwantyzacji sterowanej parametrem jakości $Q \in [1, 100]$, gdzie wyższe wartości oznaczają mniejszą stratę informacji. Przy niskich wartościach $Q$ pojawiają się charakterystyczne artefakty blokowe na granicach bloków.

\subsubsection{JPEG2000}
Format JPEG2000 wykorzystuje dyskretną transformatę falkową (DWT), operującą na całym obrazie zamiast na blokach. Eliminuje to artefakty blokowe i~umożliwia progresywną dekompresję. W~standardzie DICOM jest rekomendowanym formatem kompresji stratnej dla obrazów medycznych.

\subsubsection{AVIF}
Format AVIF bazuje na kodeku wideo AV1, wykorzystując zaawansowane techniki predykcji wewnątrzramkowej z~blokami o~zmiennym rozmiarze (od 4$\times$4 do 64$\times$64 pikseli). Oferuje znacząco lepszą efektywność kompresji w~porównaniu z~formatami poprzedniej generacji.

\subsection{Mierniki jakości obrazu}

\subsubsection{PSNR (Peak Signal-to-Noise Ratio)}
Szczytowy stosunek sygnału do szumu wyrażony w~decybelach:
\begin{equation}
    \text{PSNR} = 10 \cdot \log_{10}\left(\frac{MAX_I^2}{\text{MSE}}\right)
\end{equation}
gdzie $MAX_I$ to maksymalna wartość piksela (255 dla obrazów 8-bitowych), a~MSE to średni błąd kwadratowy między obrazem oryginalnym a~skompresowanym. Wartości powyżej 40~dB wskazują na bardzo wysoką jakość, wartości 30--40~dB na akceptowalną jakość, natomiast wartości poniżej 30~dB na istotną degradację.

\subsubsection{SSIM (Structural Similarity Index Measure)}
Wskaźnik podobieństwa strukturalnego porównujący luminancję, kontrast i~strukturę:
\begin{equation}
    \text{SSIM}(x,y) = \frac{(2\mu_x\mu_y + C_1)(2\sigma_{xy} + C_2)}{(\mu_x^2 + \mu_y^2 + C_1)(\sigma_x^2 + \sigma_y^2 + C_2)}
\end{equation}
gdzie $\mu_x$, $\mu_y$ to średnie, $\sigma_x^2$, $\sigma_y^2$ to wariancje, $\sigma_{xy}$ to kowariancja, a~$C_1$, $C_2$ to stałe stabilizacyjne. SSIM przyjmuje wartości z~przedziału $[0, 1]$, gdzie 1 oznacza identyczność obrazów.

\subsubsection{Współczynnik kompresji}
Stosunek rozmiaru pliku oryginalnego do rozmiaru pliku skompresowanego:
\begin{equation}
    CR = \frac{S_{\text{oryginalny}}}{S_{\text{skompresowany}}}
\end{equation}

\subsection{Model klasyfikacji}

Do klasyfikacji segmentów tętnic wieńcowych wykorzystano architekturę ResNet-50 \parencite{he2016deep} z~wagami wstępnie nauczonymi na zbiorze ImageNet. Warstwa wyjściowa została zastąpiona warstwą w~pełni połączoną z~26 neuronami odpowiadającymi klasom segmentów według schematu SYNTAX Score.

Dodatkowo przeprowadzono eksperymenty z~architekturą EfficientNet-B0 \parencite{tan2019efficientnet} w~celu weryfikacji uogólnialności obserwacji.

\subsection{Mierniki skuteczności klasyfikacji}

\subsubsection{Dokładność (Accuracy)}
\begin{equation}
    \text{Accuracy} = \frac{\text{TP} + \text{TN}}{\text{TP} + \text{TN} + \text{FP} + \text{FN}}
\end{equation}

\subsubsection{Miara F1}
Średnia harmoniczna precyzji i~czułości:
\begin{equation}
    F1 = 2 \cdot \frac{\text{Precision} \cdot \text{Recall}}{\text{Precision} + \text{Recall}}
\end{equation}
W~niniejszej pracy zastosowano warianty \textit{macro} (równa waga każdej klasy) oraz \textit{weighted} (waga proporcjonalna do liczebności klasy).

\subsection{Macierz pomyłek}
Macierz pomyłek (\textit{confusion matrix}) przedstawia rozkład predykcji modelu względem etykiet rzeczywistych. Dla problemu wieloklasowego (26 klas) macierz ma wymiar 26$\times$26, gdzie element $(i,j)$ oznacza liczbę próbek klasy $i$ zaklasyfikowanych jako klasa $j$.

% ===================================================================
% 7. EKSPERYMENT
% ===================================================================
\section{Eksperyment}
\label{sec:eksperyment}

\subsection{Schemat eksperymentu}

Przeprowadzono badanie składające się z~następujących etapów:

\begin{enumerate}
    \item \textbf{Przygotowanie danych} -- pobranie zbioru ARCADE (3000 obrazów PNG).
    \item \textbf{Kompresja} -- wygenerowanie skompresowanych wersji obrazów w~trzech formatach (JPEG, JPEG2000, AVIF) na sześciu poziomach jakości ($Q = 100, 85, 70, 50, 30, 10$), łącznie 54\,000 obrazów.
    \item \textbf{Pomiar jakości kompresji} -- obliczenie PSNR, SSIM i~współczynnika kompresji dla wszystkich wariantów.
    \item \textbf{Eksperyment~A} -- trening modeli na danych skompresowanych, ewaluacja na danych oryginalnych.
    \item \textbf{Eksperyment~B} -- trening modelu na danych oryginalnych, ewaluacja na danych skompresowanych.
    \item \textbf{Analiza wyników} -- porównanie formatów i~wyznaczenie optymalnego poziomu kompresji.
\end{enumerate}

\subsection{Opis danych}

Zbiór ARCADE zawiera 3000 obrazów angiografii rentgenowskiej tętnic wieńcowych o~rozdzielczości 512$\times$512 pikseli w~formacie PNG. Dane podzielone są na dwa zadania:

\begin{itemize}
    \item \textbf{Syntax} (1500 obrazów) -- klasyfikacja segmentów naczyniowych do 26 klas według schematu SYNTAX Score.
    \item \textbf{Stenosis} (1500 obrazów) -- detekcja zwężeń (klasyfikacja binarna).
\end{itemize}

Każde zadanie wykorzystuje podział: trening (1050 obrazów, 70\%), walidacja (225 obrazów, 15\%) i~test (225 obrazów, 15\%). Adnotacje zapisane są w~formacie COCO JSON.

\subsection{Procedura kompresji}

Dla każdego z~3000 oryginalnych obrazów PNG wygenerowano 18 wersji skompresowanych (3 formaty $\times$ 6 poziomów jakości). Kompresję przeprowadzono z~wykorzystaniem biblioteki Pillow (JPEG, JPEG2000) oraz wtyczki pillow-avif-plugin (AVIF). Wszystkie skompresowane obrazy zostały zapisane, a~następnie ponownie wczytane do formatu macierzy numerycznych w~celu obliczenia metryk jakości.

\subsection{Konfiguracja treningu}

Parametry treningu modeli:
\begin{itemize}
    \item Architektura: ResNet-50 (pretrained ImageNet)
    \item Rozmiar wejścia: 224$\times$224 pikseli
    \item Optymalizator: Adam ($\beta_1 = 0.9$, $\beta_2 = 0.999$, weight decay = $10^{-4}$)
    \item Szybkość uczenia: $10^{-4}$ z~harmonogramem cosine annealing
    \item Wielkość batcha: 16
    \item Liczba epok: 50 (z~early stopping, patience = 10)
    \item Normalizacja: średnia i~odchylenie standardowe ImageNet
    \item Augmentacja: zmiana rozmiaru do 224$\times$224
\end{itemize}

\subsection{Eksperyment A: Trening na danych skompresowanych}

W~Eksperymencie~A zbadano wpływ jakości danych treningowych na końcową skuteczność modelu. Dla każdego formatu kompresji i~poziomu jakości wytrenowano oddzielny model, a~ewaluację przeprowadzono na oryginalnych (nieskompresowanych) danych testowych. Pozwoliło to odpowiedzieć na pytanie: \textit{czy model jest w~stanie nauczyć się diagnostycznie istotnych cech z~obrazów o~obniżonej jakości?}

\subsection{Eksperyment B: Ewaluacja na danych skompresowanych}

W~Eksperymencie~B zbadano odporność modelu na kompresję danych wejściowych w~fazie inferencji. Model wytrenowano na oryginalnych danych PNG, a~następnie przetestowano na wszystkich wariantach skompresowanych. Scenariusz ten odpowiada sytuacji telemedicyny, gdzie obrazy przesyłane do systemu AI mogą być skompresowane w~celu redukcji przepustowości.

% ===================================================================
% 8. WYNIKI
% ===================================================================
\section{Wyniki}
\label{sec:wyniki}

\subsection{Jakość kompresji}

W~Tabeli~\ref{tab:quality_metrics} przedstawiono średnie wartości PSNR, SSIM oraz współczynnika kompresji dla poszczególnych formatów i~poziomów jakości, obliczone na podstawie 3000 obrazów zbioru ARCADE (uśrednione po obu zadaniach i~wszystkich podzbiorach).

\begin{table}[H]
\centering
\caption{Średnie metryki jakości kompresji (N = 3000 obrazów na konfigurację)}
\label{tab:quality_metrics}
\begin{tabular}{llccc}
\toprule
Format & Jakość ($Q$) & PSNR [dB] & SSIM & Wsp. kompresji \\
\midrule
JPEG & 100 & 60,19 & 0,9994 & 1,46$\times$ \\
JPEG & 85  & 44,75 & 0,9689 & 4,15$\times$ \\
JPEG & 70  & 42,99 & 0,9526 & 5,94$\times$ \\
JPEG & 50  & 38,09 & 0,9134 & 8,65$\times$ \\
JPEG & 30  & 36,68 & 0,8778 & 15,07$\times$ \\
JPEG & 10  & 32,83 & 0,7774 & 40,04$\times$ \\
\midrule
JPEG2000 & 100 & $\infty$ (bezstratna) & 1,0000 & 0,41$\times$\textsuperscript{*} \\
JPEG2000 & 85  & 39,03 & 0,9212 & 2,87$\times$ \\
JPEG2000 & 70  & 37,17 & 0,8818 & 5,78$\times$ \\
JPEG2000 & 50  & 36,11 & 0,8530 & 9,69$\times$ \\
JPEG2000 & 30  & 35,44 & 0,8371 & 13,61$\times$ \\
JPEG2000 & 10  & 34,95 & 0,8255 & 17,45$\times$ \\
\midrule
AVIF & 100 & $\infty$ (bezstratna) & 1,0000 & 1,19$\times$ \\
AVIF & 85  & 46,79 & 0,9851 & 4,35$\times$ \\
AVIF & 70  & 41,77 & 0,9558 & 8,06$\times$ \\
AVIF & 50  & 37,30 & 0,8832 & 26,13$\times$ \\
AVIF & 30  & 34,97 & 0,8213 & 75,95$\times$ \\
AVIF & 10  & 32,57 & 0,7734 & 157,93$\times$ \\
\bottomrule
\end{tabular}
\par\smallskip
\footnotesize{\textsuperscript{*}Wartość $< 1$ oznacza, że plik JPEG2000 w~trybie bezstratnym jest większy niż oryginalny PNG.}
\end{table}

Analiza wyników jakości kompresji pozwala sformułować kilka istotnych obserwacji:

\begin{enumerate}
    \item \textbf{Bezstratność}: Formaty JPEG2000 ($Q = 100$) i~AVIF ($Q = 100$) zapewniają kompresję bezstratną (PSNR = $\infty$, SSIM = 1,0). Format JPEG przy $Q = 100$ nie jest bezstratny -- osiąga PSNR = 60,19~dB, co stanowi jakość niemal nierozróżnialną, lecz nie identyczną z~oryginałem.

    \item \textbf{Efektywność kompresji AVIF}: Format AVIF osiąga najwyższe współczynniki kompresji we wszystkich przedziałach jakości. Przy $Q = 10$ współczynnik kompresji wynosi 157,93$\times$, co jest prawie czterokrotnie więcej niż JPEG (40,04$\times$) i~dziewięciokrotnie więcej niż JPEG2000 (17,45$\times$) przy porównywalnym PSNR ($\sim$33--35~dB).

    \item \textbf{Zachowanie jakości}: Przy wysokich poziomach jakości ($Q = 85$) AVIF uzyskuje najwyższy PSNR (46,79~dB) i~SSIM (0,9851), przewyższając zarówno JPEG (44,75~dB; 0,9689) jak i~JPEG2000 (39,03~dB; 0,9212).

    \item \textbf{Anomalia JPEG2000}: Format JPEG2000 w~trybie bezstratnym ($Q = 100$) generuje pliki większe niż oryginalne PNG (współczynnik kompresji 0,41$\times$), co czyni go nieopłacalnym w~tym trybie. Ponadto w~trybie stratnym JPEG2000 wykazuje nietypowo małe zróżnicowanie PSNR między $Q = 10$ (34,95~dB) a~$Q = 85$ (39,03~dB) -- jedynie 4~dB różnicy, w~porównaniu z~12~dB dla JPEG i~14~dB dla AVIF.
\end{enumerate}

W~Tabeli~\ref{tab:quality_comparison} przedstawiono porównanie formatów przy zbliżonym poziomie PSNR ($\sim$37~dB), ilustrujące różnice w~efektywności kompresji.

\begin{table}[H]
\centering
\caption{Porównanie formatów przy zbliżonym poziomie jakości (PSNR $\approx$ 37~dB)}
\label{tab:quality_comparison}
\begin{tabular}{lcccc}
\toprule
Format & $Q$ & PSNR [dB] & SSIM & Wsp. kompresji \\
\midrule
JPEG     & 50 & 38,09 & 0,9134 & 8,65$\times$ \\
JPEG2000 & 70 & 37,17 & 0,8818 & 5,78$\times$ \\
AVIF     & 50 & 37,30 & 0,8832 & 26,13$\times$ \\
\bottomrule
\end{tabular}
\par\smallskip
\footnotesize{Przy porównywalnym PSNR, AVIF osiąga trzykrotnie wyższy współczynnik kompresji niż JPEG i~4,5-krotnie wyższy niż JPEG2000.}
\end{table}

\subsection{Eksperyment A: Wpływ kompresji danych treningowych}

\textit{Eksperymenty treningowe (Eksperyment~A i~B) wymagają wielogodzinnego treningu modeli na GPU i~nie zostały jeszcze przeprowadzone. Poniżej przedstawiono przygotowane tabele do uzupełnienia wynikami.}

\begin{table}[H]
\centering
\caption{Wyniki Eksperymentu~A: trening na danych skompresowanych, test na oryginałach (zadanie Syntax, ResNet-50)}
\label{tab:exp_a}
\begin{tabular}{llccc}
\toprule
Format & $Q$ (trening) & Accuracy [\%] & F1 macro & F1 weighted \\
\midrule
Baseline (PNG) & -- & -- & -- & -- \\
\midrule
JPEG & 100 & -- & -- & -- \\
JPEG & 85  & -- & -- & -- \\
JPEG & 70  & -- & -- & -- \\
JPEG & 50  & -- & -- & -- \\
JPEG & 30  & -- & -- & -- \\
JPEG & 10  & -- & -- & -- \\
\midrule
JPEG2000 & 100 & -- & -- & -- \\
JPEG2000 & 85  & -- & -- & -- \\
JPEG2000 & 70  & -- & -- & -- \\
JPEG2000 & 50  & -- & -- & -- \\
JPEG2000 & 30  & -- & -- & -- \\
JPEG2000 & 10  & -- & -- & -- \\
\midrule
AVIF & 100 & -- & -- & -- \\
AVIF & 85  & -- & -- & -- \\
AVIF & 70  & -- & -- & -- \\
AVIF & 50  & -- & -- & -- \\
AVIF & 30  & -- & -- & -- \\
AVIF & 10  & -- & -- & -- \\
\bottomrule
\end{tabular}
\end{table}

\subsection{Eksperyment B: Odporność modelu na kompresję wejścia}

\begin{table}[H]
\centering
\caption{Wyniki Eksperymentu~B: trening na oryginałach, test na danych skompresowanych (zadanie Syntax, ResNet-50)}
\label{tab:exp_b}
\begin{tabular}{llccc}
\toprule
Format & $Q$ (test) & Accuracy [\%] & F1 macro & F1 weighted \\
\midrule
Baseline (PNG) & -- & -- & -- & -- \\
\midrule
JPEG & 100 & -- & -- & -- \\
JPEG & 85  & -- & -- & -- \\
JPEG & 70  & -- & -- & -- \\
JPEG & 50  & -- & -- & -- \\
JPEG & 30  & -- & -- & -- \\
JPEG & 10  & -- & -- & -- \\
\midrule
JPEG2000 & 100 & -- & -- & -- \\
JPEG2000 & 85  & -- & -- & -- \\
JPEG2000 & 70  & -- & -- & -- \\
JPEG2000 & 50  & -- & -- & -- \\
JPEG2000 & 30  & -- & -- & -- \\
JPEG2000 & 10  & -- & -- & -- \\
\midrule
AVIF & 100 & -- & -- & -- \\
AVIF & 85  & -- & -- & -- \\
AVIF & 70  & -- & -- & -- \\
AVIF & 50  & -- & -- & -- \\
AVIF & 30  & -- & -- & -- \\
AVIF & 10  & -- & -- & -- \\
\bottomrule
\end{tabular}
\end{table}

\subsection{Porównanie formatów kompresji}

Na podstawie uzyskanych metryk jakości kompresji (Tabela~\ref{tab:quality_metrics}) można sformułować wstępne wnioski dotyczące przydatności poszczególnych formatów w~kontekście obrazowania medycznego:

\begin{itemize}
    \item \textbf{AVIF} wykazuje najlepszy stosunek jakości do rozmiaru pliku we wszystkich testowanych przedziałach. Przy $Q = 85$ osiąga SSIM = 0,9851 przy kompresji 4,35$\times$, natomiast przy $Q = 50$ -- SSIM = 0,8832 przy kompresji 26,13$\times$.

    \item \textbf{JPEG} oferuje dobre zachowanie jakości przy umiarkowanej kompresji ($Q \geq 70$: SSIM $\geq$ 0,95), jednak przy agresywnej kompresji ($Q = 10$) SSIM spada do 0,7774, co wskazuje na istotną degradację strukturalną.

    \item \textbf{JPEG2000}, pomimo statusu standardu medycznego DICOM, wykazuje zaskakująco niską efektywność kompresji w~stosunku do pozostałych formatów. Przy $Q = 85$ osiąga jedynie SSIM = 0,9212, co jest wartością niższą niż JPEG przy $Q = 70$ (SSIM = 0,9526). Wynik ten sugeruje, że parametryzacja jakości w~bibliotece Pillow dla formatu JPEG2000 może nie odpowiadać optymalnym ustawieniom stosowanym w~profesjonalnych systemach DICOM.
\end{itemize}

Ostateczne porównanie formatów pod kątem wpływu na skuteczność modeli klasyfikacji zostanie uzupełnione po przeprowadzeniu Eksperymentów~A i~B (Tabele~\ref{tab:exp_a} i~\ref{tab:exp_b}).

% ===================================================================
% 9. WNIOSKI
% ===================================================================
\section{Wnioski}
\label{sec:wnioski}

\textit{[Wnioski zostaną sformułowane po uzyskaniu i~przeanalizowaniu wyników eksperymentów. Oczekiwane wnioski obejmują:]}

\begin{enumerate}
    \item Porównanie JPEG2000 i~JPEG w~kontekście zachowania skuteczności modeli AI -- weryfikacja hipotezy H1.
    \item Ocena przydatności formatu AVIF w~obrazowaniu medycznym -- weryfikacja hipotezy H2.
    \item Identyfikacja progowego poziomu kompresji -- weryfikacja hipotezy H3.
    \item Porównanie scenariuszy z~Eksperymentu~A i~B: czy kompresja danych treningowych jest mniej lub bardziej szkodliwa niż kompresja danych wejściowych w~fazie inferencji.
\end{enumerate}

% ===================================================================
% 10. PODSUMOWANIE
% ===================================================================
\section{Podsumowanie}
\label{sec:podsumowanie}

\textit{[Podsumowanie zostanie napisane po zakończeniu badań. Powinno obejmować:]}

\begin{itemize}
    \item Wykazanie realizacji celów pracy i~weryfikacji hipotez badawczych.
    \item Praktyczne rekomendacje dotyczące kompresji obrazów angiografii wieńcowej w~systemach PACS i~telemedycynie.
    \item Zalety proponowanego podejścia: kompleksowe porównanie trzech formatów, dwa komplementarne scenariusze eksperymentalne, duży zbiór danych (54\,000 obrazów).
    \item Ograniczenia: badanie ograniczone do jednego typu modalności (angiografia rentgenowska), dwóch architektur modeli, jednego zbioru danych.
    \item Kierunki dalszych badań: rozszerzenie na inne modalności obrazowania (MRI, CT), badanie wpływu kompresji na zadania segmentacji i~detekcji, zastosowanie adaptacyjnych strategii kompresji.
\end{itemize}

% ===================================================================
% SPIS RYSUNKÓW, TABEL, BIBLIOGRAFIA
% ===================================================================

\listoffigures
\listoftables

\printbibliography[title={Bibliografia}]

\end{document}
